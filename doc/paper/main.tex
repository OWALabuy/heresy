%%%%%%%%%%%%%%%%%%%%%%%%%%%%%%%%%%%%%%%%%%%%%%%%%%%%%%%%
%论文模板来自https://zhuanlan.zhihu.com/p/496462136
\documentclass[12pt,a4paper]{article}% 文档格式
\usepackage{ctex,hyperref}% 输出汉字
\usepackage{times}% 英文使用Times New Roman
%%%%%%%%%%%%%%%%%%%%%%%%%%%%%%%%%%%%%%%%%%%%%%%%%%%%%%%%
% \title{\fontsize{18pt}{27pt}\selectfont% 小四字号,1.5倍行距
% 	{\heiti% 黑体 
% 		一种\LaTeX 模板}}% 题目
% %%%%%%%%%%%%%%%%%%%%%%%%%%%%%%%%%%%%%%%%%%%%%%%%%%%%%%%%
% \author{\fontsize{12pt}{18pt}\selectfont% 小四字号,1.5倍行距
% 	{\fangsong% 仿宋
% 		Evildoer}\thanks{向寝室大佬膜膜膜}\\% 标题栏脚注
% 	\fontsize{10.5pt}{15.75pt}\selectfont% 五号字号,1.5倍行距
% 	{\fangsong% 仿宋
% 		(末流985~~~雾里咳血学院)}}% 作者单位,“~”表示空格
% %%%%%%%%%%%%%%%%%%%%%%%%%%%%%%%%%%%%%%%%%%%%%%%%%%%%%%%%
% \date{}% 日期(这里避免生成日期)
%%%%%%%%%%%%%%%%%%%%%%%%%%%%%%%%%%%%%%%%%%%%%%%%%%%%%%%%
\usepackage{amsmath,amsfonts,amssymb}% 为公式输入创造条件的宏包
%%%%%%%%%%%%%%%%%%%%%%%%%%%%%%%%%%%%%%%%%%%%%%%%%%%%%%%%
\usepackage{graphicx}% 图片插入宏包
\usepackage{subfigure}% 并排子图
\usepackage{float}% 浮动环境,用于调整图片位置
\usepackage[export]{adjustbox}% 防止过宽的图片
%%%%%%%%%%%%%%%%%%%%%%%%%%%%%%%%%%%%%%%%%%%%%%%%%%%%%%%%
\usepackage{bibentry}
\usepackage{natbib}% 以上2个为参考文献宏包
%%%%%%%%%%%%%%%%%%%%%%%%%%%%%%%%%%%%%%%%%%%%%%%%%%%%%%%%
\usepackage{abstract}% 两栏文档,一栏摘要及关键字宏包
\renewcommand{\abstracttextfont}{\fangsong}% 摘要内容字体为仿宋
\renewcommand{\abstractname}{\textbf{摘\quad 要}}% 更改摘要二字的样式
%%%%%%%%%%%%%%%%%%%%%%%%%%%%%%%%%%%%%%%%%%%%%%%%%%%%%%%%
\usepackage{xcolor}% 字体颜色宏包
\newcommand{\red}[1]{\textcolor[rgb]{1.00,0.00,0.00}{#1}}
\newcommand{\blue}[1]{\textcolor[rgb]{0.00,0.00,1.00}{#1}}
\newcommand{\green}[1]{\textcolor[rgb]{0.00,1.00,0.00}{#1}}
\newcommand{\darkblue}[1]
{\textcolor[rgb]{0.00,0.00,0.50}{#1}} \newcommand{\darkgreen }[1]
{\textcolor[rgb]{0.00,0.37,0.00}{#1}}
\newcommand{\darkred}[1]{\textcolor[rgb]{0.60,0.00,0.00}{#1}}
\newcommand{\brown}[1]{\textcolor[rgb]{0.50,0.30,0.00}{#1}}
\newcommand{\purple}[1]{\textcolor[rgb]{0.50,0.00,0.50}{#1}}% 为使用方便而编辑的新指令
%%%%%%%%%%%%%%%%%%%%%%%%%%%%%%%%%%%%%%%%%%%%%%%%%%%%%%%%
\usepackage{url}% 超链接
\usepackage{bm}% 加粗部分公式
\usepackage{multirow}
\usepackage{booktabs}
\usepackage{epstopdf}
\usepackage{epsfig}
\usepackage{longtable}% 长表格
\usepackage{supertabular}% 跨页表格


\usepackage{algorithm}
\usepackage{algorithmic}
\usepackage{changepage}% 换页
%%%%%%%%%%%%%%%%%%%%%%%%%%%%%%%%%%%%%%%%%%%%%%%%%%%%%%%%
\usepackage{enumerate}% 短编号
\usepackage{caption}% 设置标题
\captionsetup[figure]{name=\fontsize{10pt}{15pt}\selectfont Figure}% 设置图片编号头
\captionsetup[table]{name=\fontsize{10pt}{15pt}\selectfont Table}% 设置表格编号头
%%%%%%%%%%%%%%%%%%%%%%%%%%%%%%%%%%%%%%%%%%%%%%%%%%%%%%%%
\usepackage{indentfirst}% 中文首行缩进
\usepackage[left=2.50cm,right=2.50cm,top=2.80cm,bottom=2.50cm]{geometry}% 页边距设置
\renewcommand{\baselinestretch}{1.5}% 定义行间距(1.5)
%%%%%%%%%%%%%%%%%%%%%%%%%%%%%%%%%%%%%%%%%%%%%%%%%%%%%%%%
\usepackage{fancyhdr} %设置全文页眉、页脚的格式
\hypersetup{colorlinks=true,linkcolor=black}% 去除引用红框,改变颜色
%%%%%%%%%%%%%%%%%%%%%%%%%%%%%%%%%%%%%%%%%%%%%%%%%%%%%%%%

\begin{document}% 以下为正文内容
% \maketitle% 产生标题 标题我另外弄了 这些被注释掉的内容会在论文完成后被移除
%%%%%%%%%%%%%%%%%%%%%%%%%%%%%%%%%%%%%%%%%%%%%%%%%%%%%%%%
% \lhead{}% 页眉左边设为空
% \chead{}% 页眉中间设为空
% \rhead{}% 页眉右边设为空
% \lfoot{}% 页脚左边设为空
% \cfoot{\thepage}% 页脚中间显示页码
% \rfoot{}% 页脚右边设为空
\pagestyle{fancy}%页眉页脚 启动!!!
\fancyhf{} % 清空默认样式
%%%%%%%%%%%%%%%%%%%%%%%%%%%%%%%%%%%%%%%%%%%%%%%%%%%%%%%%
\thispagestyle{empty}%当前页面 禁用页眉页脚
\begin{center}
	\includegraphics[width=15cm]{img/logo.png} % 插入学校标志图片
	\vspace{0.5cm}

	\Huge \textbf{本科生毕业设计(论文)}\\[2cm]

	\LARGE \textbf{题目:Heresy-兼容多内核的网络代理线路管理与路由管理系统}\\[3cm]
\end{center}

\begin{flushleft}
	\large
	\begin{tabbing}
		姓\hspace{1.5em}名:\hspace{1.5em} \= 欧阳闻奕  \\[0.5cm]
		学\hspace{1.5em}号:\> 2022102069 \\[0.5cm]
		学\hspace{1.5em}院:\> 计算机与信息技术学院 \\[0.5cm]
		专业/届别:\> 软件工程/2026 届 \\[0.5cm]
		指导教师:\> 丁蕊 \\[0.5cm]
		职\hspace{1.5em}称:\> 副教授 \\[0.5cm]
	\end{tabbing}
\end{flushleft}

\newpage

% 独创性声明
\thispagestyle{empty}%当前页面 禁用页眉页脚
\begin{center}
    \large \textbf{独创性声明}
\end{center}

\setlength{\parindent}{2em} % 设置段首缩进2字符
\setlength{\parskip}{2em}  % 设置段间距
{\fangsong % 在这个环境中使用仿宋字体
本人郑重声明:所呈交的本科生毕业设计(论文),是本人在导师的指导下,独立进行研究工作所取得的成果。除文中已经注明引用的内容外,本论文不含任何其他个人或集体已经发表或撰写过的研究成果。对本文的研究做出重要贡献的个人和集体,均已在文中以明确方式标明。本声明的法律结果由本人承担。

\begin{flushright} % 设置右对齐环境
本科生毕业设计(论文)作者签名:\\
\vspace{1em} % 添加适当的空行 分隔签名和正文
\includegraphics[width=4cm]{img/owalabuy_sign.png} \\ % 插入签名图片
签字日期: 2025年1月5日
\end{flushright}}

\newpage

\fancyhf{} % 清空默认样式
\fancyhead[C]{牡丹江师范学院本科生毕业设计(论文)} % 从摘要开始定义页眉

%摘要(中英文)
\begin{abstract}
    %摘要应该在论文完成后再写 以下仅为填充 到时候再认真写awa
    %摘要的写作方法
    %目的: 研究的对象 范围 目的
    %方法: 采用了哪些手段 研究方法
    %结果: 陈述论文研究结果 新见解等
    %结论: 通过对问题的研究所得出的重要结论 主要观点 理论意义或实用价值等
	\fangsong
    网络代理是我们常用的反审查手段 目前主流的内核有xray-core v2ray-core sing-box hysteria2 clash等等 它们需要用户手动编写配置文件 较为麻烦且对用户的技术水平要求较高 所以代理线路管理软件出现了 在Windows平台和Android平台上有一些GUI客户端 如V2RayN/G Clash系 Surfboard 它们都是基于图形界面的 支持对机场\footnote{网络代理的线路供应商}的订阅链接进行解析 获取到链接中的节点\footnote{一个用于代理的服务器实例称为一个节点}信息 用户还可以手动添加节点 用户直接在GUI客户端中选择节点 进行连接
\end{abstract}

\begin{adjustwidth}{1.06cm}{1.06cm}
    \fontsize{10.5pt}{15.75pt}\selectfont{\heiti{关键词:}\fangsong{网络代理;翻墙;节点;路由;机场订阅;反向代理;xray; proxy; hysteria2; C++; Qt}}\\
\end{adjustwidth}

%中文摘要与英文摘要之间要不要换页呢 到时候再说吧...
\begin{center}% 居中处理
	{\textbf{Abstract}}% 英文摘要
\end{center}
\begin{adjustwidth}{1.06cm}{1.06cm}% 英文摘要内容
	\hspace{1.5em}Attention!If you input "dif{}ferent", the computer will output "different", but if you input "dif\{\}ferent", the computer will output "dif{}ferent"
\end{adjustwidth}

\newpage% 从新的一页继续

%目录
\tableofcontents
\newpage

% 正文内容
\fancyfoot[C]{\thepage} % 页脚中间显示页码
\pagenumbering{arabic} % 重置页码为阿拉伯数字
\setcounter{page}{1} % 从第 1 页开始计数

\section{系统概述}
\subsection{概述}
\subsection{课题意义}
\subsection{主要内容}

\newpage

\section{系统开发环境}
\subsection{GNU/Linux系统}
\subsection{C++语言}
\subsection{Vim编辑器}
\subsection{Qt框架}
\subsection{Sqlite3数据库}

\newpage

\section{系统分析}
\subsection{可行性分析}
\subsection{需求分析}
\subsection{性能分析}

\newpage

\section{系统设计}
\subsection{系统架构}
\subsection{系统E-R图}
\subsection{数据库设计}
\subsubsection{数据库实体}
\subsubsection{数据库设计表}

\newpage

\section{详细设计与实现}

\newpage

\section{系统测试}
\subsection{系统测试的目的与方法}
\subsection{测试用例与结果}
\subsection{测试总结}

% 总结
\section*{总结}%无编号章节
\addcontentsline{toc}{section}{总结} % 添加到目录
总结内容写在这里。

% 致谢
\section*{致谢}
\addcontentsline{toc}{section}{致谢} % 添加到目录
致谢内容写在这里。

%\section{摆烂二阶段}
%\subsection{摆的理论基础}
%\subsubsection{Evildoer的摆理论}
%大本钟下寄快递,上面开摆下面寄
%\subsubsection{摆理论的完善与发展}
%\subsection{摆的实际应用}
%\begin{enumerate}[1.]% 列举时编号
%	\item 啊对
%	      \begin{enumerate}[(a)]% 次级序号
%		      \item 太对辣
%		      \item 好对捏
%	      \end{enumerate}
%	\item 啊对对
%	\item 啊对对对\footnote{变成光守护麻衣学姐}% 脚注
%\end{enumerate}
%
%\section{摆烂三阶段}
%至臻无双
%
\newpage
\begin{thebibliography}{99}% 参考文献,{}内表示序号最大位数(两位)
	\bibitem{ref1}Evildoer. 开摆的深刻内涵[J]. 大学物理, 11.4(2022):1-4.
	\bibitem{ref2}Propht Joseph. 摆王的自我修养[M]. Supercell出版社, 01(2333):-2-$-\infty$.
\end{thebibliography}
\end{document}
